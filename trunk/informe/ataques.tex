\section{Ataques}

La virtualizaci'on caus'o un gran impacto al brindar un nuevo paradigma sobre la forma en la cual interactuamos con equipos, realizamos pruebas, administramos servidores, entre otros. Y para la segurdad inform'atica representa un nuevo conjunto de herramientas para el an'alisis de programas maliciosos y disminuir la exposici'on de sistemas reales.

Pero para lograrlo se necesita un atributo importante: la \negrita{isolaci'on} de las m'aquinas virtuales del mundo exterior, f'isico y real. Es decir, cualquier proceso (incluyendo al sistema operativo) corriendo en una m'aquina virtual no deber'ia poder saber que lo est'a. Menos a'un deber'ia poder interactual con el sistema operativo host o las dem'as m'aquinas virtuales. 'Esta cualidad no siempre es conseguida, y existen varios vectores de ataque disponibles para que una aplicaci'on maliciosa logre romper la isolaci'on.

A continuaci'on mostraremos los puntos clave donde se puede atacar un ambiente virtualizado y formas con las cuales prevenir de ser vulnerados por 'estos. Cabe aclarar que algunos de los puntos nombrados pueden ser considerados como caracter'isticas favorables de los ambientes virtualizados. Pero tambi'en pueden ser de ayuda para un atacante.

Los ataques conseguidos en la pr'actica utilizando las t'ecnicas aqu'i presentadas ser'an explicados en las siguientes secci'ones.

\subsection{Detecci'on de ambiente virtualizado}

El objetivo de una m'aquina virtual es proveerle al usuario un equipo que parece real pero en realidad no lo es. 'Esto aplica principalmente para las aplicaciones que ejecute. Por eso se espera que cada una de 'estas se comporten dentro de una VM de la misma forma que se comportar'ia dentro de de un equipo normal. 

Consiguiendo tal objetivo uno puede analizar el consumo de recursos de la aplicaci'on y las entradas y salidas que realiza tales como conexiones al exterior, accesos a archivos, estructuras de datos del sistema (como el registro en Windows), etc con total certeza que los datos tomados hubiesen sido similares en ambientes reales. Tambi'en uno podr'ia ejecutar programas potencialmente maliciosos en entornos seguros (m'aquinas virtuales comport'andose como sandboxes) y ver si realmente lo son; o auditar los da�os que realiza si se sabe que efectivamente lo es y encontrar un m'etodo de protecci'on adecuado.

Especialmente en el caso de los programas maliciosos es cuando se hace indispensable la necesidad de no revelar que se encuentra en un equipo virtualizado. De no lograrlo, 'este podr'ia comportarse de forma diferente (no maliciosa) ya que sospecha que est'a siendo analizado.

Lamentablemente los equipos con vitualizaci'on completa resultan p'oco 'utiles en la pr'actica debido a la gran ineficiencia de recursos que se genera y es por 'esto que la mayor'ia de las soluciones de virtualizaci'on utilizan paravirtualizaci'on. 'Este enfoque presenta modificaciones al kernell del sistema operativo guest con lo cual ya se compromete desde el inicio la transparencia de la m'aquina virtual. Por su parte, el hypervisor y las VMM no dejan de ser programas y como tales pueden contener vunerabilidades a la hora de esconder su verdadera naturaleza a los procesos corriendo dentro. Adem'as un ambiente virtualizado puede poseer una mala configuraci'on por parte de sus administradores que vulneren al sistema. As'i los creadores de c'odigos maliciosos obtienen herramientas para detectar ambientes virtualizados. Las m'as importantes se explicar'an en detalle en la secci'on correspondiente.

\subsubsection{Ataques de tiempo}

Existen diferencias temporales en ciertas operaciones cuando se trabaja con hardware real y uno virtualizado. Por ejemplo, leer un registro PCI puede tardar cientos de ciclos en leer desde el hardware pero solo uno cuando 'este ya se encuentra en un registro del CPU. 'Estas diferencias pueden no apreciarse si se corre en un ambiente virtualizado, ya que el hypervisor genera nuevas instrucciones que pueden alterar los datos de los registros del procesador. Un programa podr'ia utilizar las diferencias esperadas para detectar un ambiente virtualizado.

\subsubsection{Invirtiendo el enfoque de la defensa}

Dada la dificultad de esconder aquella informaci'on que advierte si el ambiente es virtualizado o no, los programas maliciosos poseen mejores herramientas para la detecci'on de 'estos y comportarse de maneras diferentes a como lo har'ian en equipos reales. As'i es como la ingenier'ia inversa que se puede realizar sobre 'estos programas puede quedar resultar in'util en varios casos. 'Esto presenta un serio desaf'io para los encargados de la protecci'on de sistemas productivos. 

Pero existe otro enfoque y es el de, en lugar de hacer ver a una m'aquina virtual como f'isica, hacer ver a una m'aquina f'isica como virtual. As'i se logra que todo programa malicioso que infecta a un equipo productivo entre en un estado de paranoia constante al considerar que lo est'an analizando e intentando de desensamblar. Luego, 'este se comportar'ia ben'eficamente y nunca logre da�ar ni infectar al sistema\footnote{M'as informaci'on: \url{http://www.eecs.umich.edu/?zmao/Papers/DCCS-xu-chen.pdf}}.

\subsection{VM Escape}

Como ya se dijo, una aplicaci'on maliciosa posee ciertos medios para detectar ambientes virtualizados y comportarse de manera diferente si sospecha que est'a siendo analizado. Pero tambi'en puede aprovecharse a'un mas de las vulnerabilidades encontradas y causar da�os en el mismo ambiente.

'Estos ataques se llaman VM Escape y por lo general consisten en explotar alguna vulnerabilidad en el hypervisor (principalmente, buffer overflow) y conseguir que 'ejecute c'odigo arbitrario. Como 'este corre en el sistema operativo host, el c'odigo maliciosos afectar'a al sistema operativo host y se ejecutar'a con los privilegios del hypervisor (por lo general son elevados debido a las tareas de bajo nivel que requiere efectuar en el sistema). 

As'i se compromete al equipo f'isico y a todas las demas m'aquinas virtuales ejecutando en 'el ya que le hypervisor posee control absoluto sobre 'estas. Es decir, un programa malicioso ejecutado en una m'aquina podr'ia apropiarse de todo el sistema. 

\subsection{Comunicaci'on entre VMs y entre guest y host}

Dada la necesidad de isolaci'on ya explicada anteriormente, una m'aquina virtual (es decir, ningun proceso ejecut'andose en ella) no deber'ia poder comunicarse con las dem'as corriendo en el mismo equipo f'isico o con el host. Pero as'i como se dijo antes el hypervisor puede poseer vulnerabilidades que comprometan 'este requerimiento y brinden puertas de acceso a los programas para que monitoreen y alteren la informaci'on de ellos.

\subsubsection{Medios comunes}

Una forma son los medios comunes brindados adrede por el software de virtualizaci'on utilizado. Los m'as comunes son

\begin{description}

\item[Clipboard] Algunos productos de virtualizaci'on permiten que el host y los guests compartan el clipboard (o portapapaeles) pudiendo cada uno de ellos leer y modificar sus contenidos. En algunos casos la informaci'on que se aloje en 'este puede ser confidencial, como una contrase�a, y no deber'ia ser compartida con los dem'as equipos. Adem'as, al ser un medio donde se puede alojar cualquier tipo de informaci'on, se podr'ia llegar a utilizar como covert channel para facilitar un ataque desde una m'aquina virtual a otra (o al host). 

\item[Shares] Algunos productos tambi'en traen por defecto carpetas compartidas por cada guest y el host con el objeto de facilitar el tr'afico de archivos entre 'estos. Igual que con el clipboard, puede ser un aspecto no deseado y deber'ia deshabilitarse de ser as'i. 

\item[Integraci'on de aplicaciones] Las 'ultimas tecnolog'ias de virtualizaci'on pretenden facilitar la operatoria del usuario final y buscan obtener m'aquinas virtuales m'as integradas entre s'i y el host a mayores niveles que los mostrados anteriormente. Por ejemplo, VMWare posee el modo Fusion\footnote{M'as informacion sobre VMWare Fusion: \url{http://www.vmware.com/products/fusion}} que permite ejecutar las aplicaciones del guest como si corriesen en el host, brind'andole acceso a recursos sensibles como el sistema de archivos. Por supuesto, en alguntos casos son caracter'isticas deseables, pero en otros no y se debe tener cuidado a la hora de configurarlas para obtener el nivel de isolaci'on requerido.

\end{description}

\subsubsection{Redes virtuales}

Los hypervisor por lo general utiizan hubs o switches virtuales para responder a las necesidades de red de las m'aquinas virtuales. Bajo 'estos enfoques el host ser'ia un gateway entre la red virtual y las redes externas. 

En las redes reales existen ataques tales como t'ecnicas para leer la informaci'on que circula en las mismas (sniffing) y hacerse pasar por otro equipo (spoofing) aprovechando la ubicaci'on de un equipo en la misma red que otro. Respecto a la primera, con un hub es trivial ya que los paquetes son enviados a todos los equipos en la misma red, y con un switch se pueden utilizar ataques de ARP Poisoning\footnote{M'as informaci'on sobre ARP Poisoning: \url{http://en.wikipedia.org/wiki/Arp_poisoning}}. Sobre la segunda, principalmente consiste en utilizaci'on de direcciones MAC de otros equipos o incluso tambi'en con ARP Poisoning\footnote{M'as informaci'on sobre Spoofing: \url{http://en.wikipedia.org/wiki/Spoofing_attack}}.

Los hubs y switches virtuales se comportan de la misma forma que sus contrapartes f'isicas, por lo que son vulnerables a los mismos ataques. Por lo tanto si no se tiene cuidado en la configuraci'on de la red vitual una m'aquina virtual podr'ia monitorear informaci'on o hacerse pasar por otra de ellas utilizando 'este medio de comunicaci'on.

\subsubsection{Monitoreo de host a guest}

El hypervisor posee control total sobre las m'aquinas virtuales que ejecuta. Puede brindarle o quitarle recursos, ejecutarlas, reiniciarlas o apagarlas, etc. Pero tambi'en puede monitorear la actividad que realizan, ya sea la comunicaci'on que circula por la red virtual (como se dijo, el host hace las veces de un gateway de 'esta), los archivos que accede (junto a la informaci'on que lee y escribe de 'estos) o la memoria que utilizan {\it en caliente}, es decir, en uso.

'Esto presenta un riesgo para la informaci'on encontrada dentro de las m'aquinas virtuales. Principalmente no porque el hypervisor sea malicioso y snifee 'esta con fines mal'eficos (aunque podr'ia serlo) si no por el peligro que 'este sea comprometido debido a un programa que s'i posea fines malvados, como puede ser un virus, o un usuario explotando una vulnerabilidad en 'el. 

Y puede resultar en un compromiso mayor a los que suceden en ambientes no virtualizados. Debido a las t'ecnicas de memoria segmentada que presentan la mayor'ia de los sistemas operativos hoy en d'ia, es casi imposible que un proceso malicioso accese a la informaci'on que maneja en 'ese momento otro proceso; es decir, no puede leer memoria que no sea suya. Pero si el proceso malicioso vulnerase al hipervisor, encontrar'ia la gran ventaja de conseguir 'este tipo de monitoreo por sobre uno ejecut'andose en un guest.

Como cualquier programa vulnerable, lo mejor es aplicarle al hypervisor los parches a la brevedad posible para evitar 'este tipo de problemas y mantener todo el sistema operativo host y la red a la cual se encuentra conectado en condiciones 'optimas para cumplir con los requerimientos de seguridad del sistema.


\subsection{Denegaci'on de servicio}

El host comparte sus recursos con las m'aquinas virtuales que ejecuta. Por ello es importante que el hypervisor asegure una utilizaci'on justa de los mismos. De no lograrlo un guest podr'ia causar una utilizaci'on masiva de cierto recurso (principalmente red, file system, memoria o procesador) y causar starvation, lo que se traduce en un ataque de denegaci'on de servicio sobre el host y las dem'as m'aquinas virtuales que 'este ejecuta.


\subsection{Modificaci'ones al hypervisor y m'aquinas virtuales}

Como ya se dijo, el hypervisor posee control absoluto sobre las m'aquinas virtuales. Si 'este se compromete y un atacante logra que ejecute c'odgo arbitrario todo 'ese control estar'ia en sus manos. Pero tambi'en se podr'ia modificar al hypervisor para se comporte de la forma que el intruso desee. Por ejemplo, favorecer a alguna m'aquina virtual en la asignaci'on de recursos, permitiendo que s'olo corran alguna de ellas, gener'andoles comportamientos err'aticos causando DoS, etc.  Por eso se debe tener gran cuidado en quien accede a su binario y librer'ias que utiliza.

Pero las modificaciones tambi'en pueden realizarse al nivel de la m'aquina virtual. Lo que se comprometer'ia aqu'i ser'ian los programas e informaci'on que posee, haciendo que el atacante agregue el comportamiento que desee a 'estos o al sistema operativo guest. Nuevamente, se debe controlar el acceso a los archivos de la la m'aquina virtual.


